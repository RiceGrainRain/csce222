\documentclass{article}
\usepackage{amsmath,amssymb,amsthm,latexsym,paralist}
\usepackage{fancyhdr}

\theoremstyle{definition}
\newtheorem{problem}{Problem}
\newtheorem*{solution}{Solution}
\newtheorem*{resources}{Resources}

\newcommand{\name}[2]{\noindent\textbf{Name: #1}\hfill \textbf{UIN: #2}
  \newcommand{\myName}{#1}
  \newcommand{\myUIN}{#2}
}

\newcommand{\honor}{\noindent On my honor, as an Aggie, I have neither
  given nor received any unauthorized aid on any portion of the
  academic work included in this assignment. Furthermore, I have
  disclosed all resources (people, books, web sites, etc.) that have
  been used to answer this homework. \\[2ex]
 \textbf{Electronic signature: \underline{ (Manas Navale) } } } % type your full name here
 
\newcommand{\checklist}{\noindent\textbf{Checklist:}
\begin{compactitem}[$\Box$] 
\item Did you type in your name and UIN? 
\item Did you disclose all resources that you have used? \\
(This includes all people, books, websites, etc.\ that you have consulted)
\item Did you sign that you followed the Aggie Honor Code? 
\item Did you solve all problems? 
\item Did you submit both the .tex and .pdf files of your homework to each correct link on Canvas? 
\end{compactitem}
}

\newcommand{\problemset}[1]{\begin{center}\textbf{Problem Set #1}\end{center}}
\newcommand{\duedate}[1]{\begin{quote}\textbf{Due dates:} Electronic
    submission of \textsl{yourLastName-yourFirstName-hw5.tex} and 
    \textsl{yourLastName-yourFirstName-hw5.pdf} files of this homework is due on
    \textbf{#1} on \texttt{https://canvas.tamu.edu}. You will see two separate links
    to turn in the .tex file and the .pdf file separately. Please do not archive or compress the files.  
    \textbf{If any of the two files are missing, you will receive zero points for this homework.}\end{quote} }

\newcommand{\N}{\mathbf{N}}
\newcommand{\R}{\mathbf{R}}
\newcommand{\Z}{\mathbf{Z}}

\fancyhead[L]{\myName}
\fancyhead[R]{\myUIN}
\pagestyle{fancy}

\begin{document}
\begin{center}
{\large
CSCE 222 Discrete Structures for Computing -- Fall 2023\\[.5ex]
Hyunyoung Lee\\}
\end{center}
\problemset{5}
\duedate{Monday, 10/23/2023 before 11:59 p.m.}
\name{ (Manas Navale) }{ (333006797) } % Type your first and last name and UIN here
% Omit the parentheses surrounding name and UIN.
% Your name should include your first and last names. 
% Your name and UIN that you type in here are propagated by LaTeX 
% to the header part of each page on the PDF output automatically.

\begin{resources} (All people, books, articles, web pages, etc.\ that
  have been consulted when producing your answers to this homework)
\end{resources}
\honor

\bigskip

\noindent
Total $100+10$ (bonus) points.

\bigskip

\noindent
The intended formatting is that this first page is a cover page and each 
problem solved on a new page. You only need to fill in your solution between 
the \verb|\begin{solution}| and \verb|\end{solution}| environment.  
Please do not change this overall formatting.

\bigskip

\vfill
\checklist

\newpage
\begin{problem} (10 points) Section 11.1, Exercise 11.3 
\end{problem}
\begin{solution}

~\\
~\\
We have f(n) = $n^2 + 2n$ and g(n) = $n^2$.

Now, we want to see if the ratio of f(n) to g(n) approaches 1 as n goes to infinity. In other words, we're looking at $\lim_{n\to\infty} \frac{f(n)}{g(n)}$.

Let's simplify this ratio:
\[
\lim_{n\to\infty} \frac{n^2 + 2n}{n^2}
\]

Now, we can factor out $n^2$ from both the numerator and denominator:
\[
\lim_{n\to\infty} \frac{n^2(1 + 2/n)}{n^2}
\]

We can cancel out $n^2$ from the top and bottom:
\[
\lim_{n\to\infty} \frac{1 + 2/n}{1}
\]

Now, as n goes to infinity, the term $2/n$ approaches 0 because any number divided by infinity is essentially zero:
\[
\lim_{n\to\infty} \frac{1 + 2/n}{1} = 1 + 2 \cdot 0 = 1
\]

So, when n gets really large, the ratio of f(n) to g(n) is 1. This means that f(n) behaves very similarly to g(n) as n becomes large, and we can say that $f(n) \sim g(n)$.
\end{solution}

\newpage
\begin{problem} (20 points) Section 11.3, Exercise 11.14.
[Requirement: Study the definition of $\asymp$ involving the inequalities carefully 
and use the definition to answer the questions.] 
\end{problem}
\begin{solution}
~\\
~\\  
(i) $f \asymp f$:
To show that a function is equivalent to itself, it is sufficient to demonstrate that the limit of the quotient of the two functions is equal to 1 as the input approaches infinity. Formally, we need to show that:
\[
\lim_{{n \to \infty}} \frac{{f(n)}}{{f(n)}} = 1
\]

(ii) $f \asymp g$ if and only if $g \asymp f$:
The equivalence of functions is symmetric, so if $f \asymp g$, then $g \asymp f$. To prove the "if and only if" part, we need to show that if $g \sim f$, then $f \sim g$.
\[
\lim_{{n \to \infty}} \frac{{g(n)}}{{f(n)}} = 1
\]

(iii) $f \asymp g$ and $g \asymp h$ implies $f \asymp h$:
To prove this, we can use the transitive property of equivalence. If $f \asymp g$ and $g \asymp h$, it means:
\[
\lim_{{n \to \infty}} \frac{{g(n)}}{{f(n)}} = 1
\]
and
\[
\lim_{{n \to \infty}} \frac{{h(n)}}{{g(n)}} = 1
\]
We want to show that $f \sim h$, which means:
\[
\lim_{{n \to \infty}} \frac{{h(n)}}{{f(n)}} = 1
\]
\end{solution}

\newpage
\begin{problem} (15 points)
Prove that $3n^2+41 \in O(n^3)$ by giving a direct proof based on the definition of 
big-$O$ involving the inequalities and absolute values, as given in the lecture notes 
Section 11.4.  

To do so, first write out what $3n^2+41 \in O(n^3)$ means according to the definition.  
Then, you need to find a positive real constant $C$ and a positive integer $n_0$ 
that satisfy the definition.
\end{problem}
\begin{solution}
  ~\\
  ~\\
  To prove that $3n^2 + 41 \in O(n^3)$, we need to show that there exist positive real constants $C$ and $n_0$ such that for all $n \geq n_0$, the following inequality holds:

\[
0 \leq 3n^2 + 41 \leq Cn^3
\]

To find suitable values for $C$ and $n_0$, we consider the inequality $0 \leq 3n^2 + 41 \leq Cn^3$. For $n \geq 1$, we have:

\[
3n^2 + 41 \leq 3n^3 + 41n^2 \leq 44n^3
\]

To make this inequality valid, we choose $C = 44$. Now, we need to find an $n_0$ such that:

\[
3n^3 + 41n^2 \leq 44n^3 \quad \text{for all } n \geq n_0
\]

Simplifying the inequality:

\[
3 + \frac{41}{n} \leq 44
\]

Solving for $n$ in the middle term:

\[
\frac{41}{n} \leq 44 - 3 = 41
\]

This simplifies to:

\[
1 \leq n
\]

Thus, for $n_0 = 1$ and $C = 44$, the inequality $3n^2 + 41 \leq 44n^3$ holds for all $n \geq 1$. Therefore, $3n^2 + 41 \in O(n^3)$.
\end{solution}

\newpage
\begin{problem} (15 points) 
Prove that $\frac{1}{2}n^2+5 \in \Omega(n)$ by giving a direct proof based on the 
definition of big-$\Omega$ involving the inequalities and absolute values, as given 
in the lecture notes Section 11.5.

To do so, first write out what $\frac{1}{2}n^2+5 \in \Omega(n)$ means according 
to the definition. Then, you need to find a positive real constant $c$ and a positive 
integer $n_0$ that satisfy the definition.
\end{problem}
\begin{solution}

~\\
~\\  
We want to prove that $\frac{1}{2}n^2 + 5$ grows at least as fast as $n$ as $n$ becomes large.

According to the definition of big-$\Omega$, we need to find positive constants $c$ and $n_0$ such that the inequality $0 \leq c \cdot n \leq \frac{1}{2}n^2 + 5$ holds for all $n$ greater than or equal to $n_0$.

For $n \geq 1$, we observe that $c \cdot n$ is always less than or equal to $\frac{1}{2}n^2$.

So, we set up the inequality as follows:

\[
c \cdot n \leq \frac{1}{2}n^2 \leq \frac{1}{2}n^2 + 5
\]

To make this inequality true, we can choose $c = \frac{1}{2}$. Now, we need to find an appropriate $n_0$.

Simplifying, we find that $n \leq 1$ ensures the inequality holds for all $n \geq n_0$.

Since $n$ is always greater than or equal to 1, the inequality holds for all $n \geq 1$.

Therefore, we've shown that $\frac{1}{2}n^2 + 5$ is indeed in $\Omega(n)$ by choosing suitable values for $c$ and $n_0$ based on the definition.
\end{solution}

\newpage
\begin{problem} ($10+10=20$ points) Read Section 11.6 carefully before attempting 
this problem.

Analyze the running time of the following algorithm using a step count analysis 
as shown in the Horner scheme (Example 11.40).  
\begin{verbatim}
// search a key in an array a[1..n] of length n
search(a, n, key)        cost   times
  for k in (1..n) do      c1    [n]   
    if a[k]=key then      c2    [n]
       return k           c3    [1]
  endfor                  c4    [1]
  return false            c5    [1]
\end{verbatim}
(a) Fill in the \verb|[  ]|s in the above code each with a number or an expression involving
\verb|n| that expresses the step count for the line of code.

\medskip
\noindent
(b) Determine the worst-case complexity of this algorithm and give it in the $\Theta$ notation.
Show your work and explain using the definition of $\Theta$ involving the inequalities. 
\end{problem}
\begin{solution} (For part (b))
  ~\\
  ~\\
  The worst-case scenario occurs when the key is not found in the array, and the algorithm has to iterate through the entire array.
  ~\\
  ~\\
  In this case, the for loop will run n times, where n is the length of the array.
  Therefore, the worst-case complexity of this algorithm is $O(n)$, as the number of steps is directly proportional to the length of the array. Final Answer: The worst-case complexity of this algorithm is $O(n)$.
\end{solution}

\newpage
\begin{problem} ($15+15=30$ points) Read Section 11.6 carefully before attempting this problem.
Analyze the running time of the following algorithm using a step count analysis 
as shown in the Horner scheme (Example 11.40).
\begin{verbatim}
// determine the number of digits of an integer n
binary_digits(n)            cost  times
  int cnt = 1                c1   [1]
  while (n > 1) do           c2   [depends on n]
    cnt = cnt + 1            c3   [1]
    n = floor( n/2.0 )       c4   [1]
  endwhile                   c5   [depends on n]
  return cnt                 c6   [1]
\end{verbatim}
\noindent
(a) Fill in the \verb|[  ]|s in the above code each with a number or an expression involving
\verb|n| that expresses the step count for the line of code.

\medskip
\noindent
(b) Determine the worst-case complexity of this algorithm as a function of $n$
and give it in the $\Theta$ notation.
Show your work and explain using the definition of $\Theta$ involving the inequalities. 
\end{problem}
\begin{solution} (For part (b))
\end{solution}
\textbf{(b)} 
To determine the worst-case complexity of this algorithm, we need to understand how many times the while loop iterates in the worst case. 
The loop continues until the value of 'n' becomes less than or equal to 1. This can be found by dividing 'n' by 2 
repeatedly until it reaches 1 or less. In other words, we want to find how many times we can divide 'n' by 2 until it's less than or equal to 1.
Mathematically, we express this as:

\[
n \leq 2^k
\]

To find 'k,' we take the base-2 logarithm of both sides:

\[
\log_2(n) \leq k
\]

This means the number of iterations 'k' required for the while loop to terminate is bounded by the base-2 logarithm of 'n.' Therefore, the worst-case time complexity of this algorithm is approximately proportional to the logarithm of 'n,' which is often written as $\Theta(\log_2(n))$.

In simpler terms, the time it takes to find the number of binary digits in 'n' is closely related to the logarithm of 'n,' making the algorithm's complexity $\Theta(\log_2(n))$.
\end{document}
