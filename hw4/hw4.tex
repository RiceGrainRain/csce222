\documentclass{article}
\usepackage{amsmath,amssymb,amsthm,latexsym,paralist}
\usepackage{fancyhdr}

\theoremstyle{definition}
\newtheorem{problem}{Problem}
\newtheorem*{solution}{Solution}
\newtheorem*{resources}{Resources}

\newcommand{\name}[2]{\noindent\textbf{Name: #1}\hfill \textbf{UIN: #2}
  \newcommand{\myName}{#1}
  \newcommand{\myUIN}{#2}
}
\newcommand{\honor}{\noindent On my honor, as an Aggie, I have neither
  given nor received any unauthorized aid on any portion of the
  academic work included in this assignment. Furthermore, I have
  disclosed all resources (people, books, web sites, etc.) that have
  been used to answer this homework. \\[2ex]
  \textbf{Electronic signature: \underline{ (Manas Navale) } } } % type your full name here

\newcommand{\checklist}{\noindent\textbf{Checklist:}
  \begin{compactitem}[$\Box$]
    \item Did you type in your name and UIN?
    \item Did you disclose all resources that you have used? \\ (This includes all
    people, books, websites, etc.\ that you have consulted)
    \item Did you sign that you followed the Aggie Honor Code?
    \item Did you solve all problems?
    \item Did you submit both the .tex and .pdf files of your homework to each correct
    link on Canvas?
  \end{compactitem}
}

\newcommand{\problemset}[1]{\begin{center}\textbf{Problem Set #1}\end{center}}
\newcommand{\duedate}[1]{\begin{quote}\textbf{Due dates:} Electronic
    submission of \textsl{yourLastName-yourFirstName-hw4.tex} and
    \textsl{yourLastName-yourFirstName-hw4.pdf} files of this homework is due on
    \textbf{#1} on \texttt{https://canvas.tamu.edu}. You will see two separate links
    to turn in the .tex file and the .pdf file separately. Please do not archive or compress the files.
    \textbf{If any of the two files are missing, you will receive zero points for this homework.}\end{quote} }

\newcommand{\N}{\mathbf{N}}
\newcommand{\R}{\mathbf{R}}
\newcommand{\Z}{\mathbf{Z}}

\fancyhead[L]{\myName}
\fancyhead[R]{\myUIN}
\pagestyle{fancy}

\begin{document}
\begin{center}
  {\large
    CSCE 222 Discrete Structures for Computing -- Fall 2023\\[.5ex]
    Hyunyoung Lee\\}
\end{center}
\problemset{4}
\duedate{Friday, 10/13/2023 before 11:59 p.m.}
\name{ (Manas Navale) }{ (333006797) } % Type your first and last name and UIN here
% Omit the parentheses surrounding name and UIN.
% Your name should include your first and last names. 
% Your name and UIN that you type in here are propagated by LaTeX 
% to the header part of each page on the PDF output automatically.

\begin{resources} (All people, books, articles, web pages, etc.\ that
  have been consulted when producing your answers to this homework)
\end{resources}
\honor

\bigskip

\noindent
Total $100+5$ (bonus) points.

\bigskip

\noindent
The intended formatting is that this first page is a cover page and each
problem solved on a new page. You only need to fill in your solution between
the \verb|\begin{solution}| and \verb|\end{solution}| environment.
Please do not change this overall formatting.

\bigskip

\noindent
\textbf{Make sure that you strictly follow the structure of induction proof as shown in the
  lecture notes and how I solved in my videos.}

\vfill
\checklist

\newpage
\begin{problem} (15 points) Section 4.1, Exercise 4.3
\end{problem}
\begin{solution}
  ~\\
  ~\\
  Base case:
  \[
    S(1) = 1^2 = 1\] and Also \[1 * (1+1) * 3/6
  \]
  meaning the base case is satisfied. Induction Hypothesis would be true for all
  values of $n < m$ By definition
  \[
    S(m) = S(m - 1) + m^2
  \]
  and then by the inductive step
  \[
    S(m - 1) = (m - 1) * (m - 1 + 1) * (2(m - 1)+1)/6
  \]
  Since induction step allows us to assume that the step is true for all values
  of $n < m$. Specifically n = m - 1 We have:
  \[
    S(m) = \frac{(m - 1) \cdot (m) \cdot (2m - 1)}{6} + m^2
  \]
  By rearranging terms, we get:
  \[
    S(m) = \frac{m}{6} \left( (m - 1) \cdot (2m - 1) + 6m \right)
  \]
  That is,
  \[
    S(m) = \frac{m}{6} \left( 2m^2 - 2m - m + 1 + 6m \right)
  \]
  Simplifying further:
  \[
    S(m) = \frac{m}{6} \left( 2m^2 + 3m + 1 \right)
  \]
  And finally,
  \[
    S(m) = \frac{m}{6} (m+1)(2m + 1)
  \]
  Therefore, the hypothesis is true for \(m\).

\end{solution}
\newpage
\begin{problem} (15 points) Section 4.1, Exercise 4.4
\end{problem}
\begin{solution}
  ~\\
  ~\\
  Base case:
  For \(n = 1\), we have:
  \[
    1^3 = \left(\frac{1(1+1)}{2}\right)^2
  \]
  This simplifies to \(1 = 1\), which is true. ~\\ ~\\ Inductive Hypothesis:
  Assume that the formula holds for some positive integer \(k\):
  \[
    1^3 + 2^3 + 3^3 + \ldots + k^3 = \left(\frac{k(k+1)}{2}\right)^2
  \]
  \text{Simplify:}
  \[
    1^3 + 2^3 + 3^3 + \ldots + k^3 + (k+1)^3 = \left(\frac{k(k+1)}{2}\right)^2 + (k+1)^3
  \]
  \text{Factor:}
  \[
    \frac{k^2(k+1)^2}{4} + \frac{4(k+1)^3}{4} = \frac{k^2(k+1)^2 + 4(k+1)^3}{4}
  \]
  \text{Factor Out a Common Factor:}
  \[
    (k+1)^2\left(\frac{k^2 + 4(k+1)}{4}\right)
  \]
  \text{Simplify the Expression Inside the Parentheses:}
  \[
    (k+1)^2\left(\frac{k^2 + 4k + 4}{4}\right) = (k+1)^2\left(\frac{(k+2)^2}{4}\right)
  \]
  \text{Further Simplify:}
  \[
    (k+1)^2\left(\frac{(k+2)^2}{4}\right) = \left(\frac{(k+1)(k+2)}{2}\right)^2
  \]

  Based on this we have shown that the sum of the cubes of the first $(n)$
  natural numbers is equal to $\left(\frac{n(n+1)}{2}\right)^2$.
\end{solution}

\newpage
\begin{problem} (15 points) Section 4.1, Exercise 4.5
\end{problem}
\begin{solution}
  ~\\
  ~\\
  Base Case:
  For \(n = 1\), we have:
  \[
    1^2 = \frac{1}{3}(4(1^3) - 1) = \frac{1}{3}(4 - 1) = \frac{1}{3}(3) = 1
  \]
  This is true for the base case. ~\\ ~\\ Inductive Hypothesis: Assume that the
  formula holds for some positive integer \(k\):
  \[
    1^2 + 3^2 + 5^2 + \ldots + (2k-1)^2 = \frac{1}{3}(4k^3 - k)
  \]
  ~\\
  ~\\
  Inductive Step
  We want to prove for $(k + 1)$:
  \text{Assume Hypothesis:}
  \[
    1^2 + 3^2 + 5^2 + \ldots + (2k-1)^2 + (2(k+1)-1)^2 = \frac{1}{3}(4(k+1)^3 - (k+1))
  \]

  \text{Simplify:}
  \[
    1^2 + 3^2 + 5^2 + \ldots + (2k-1)^2 + (2(k+1)-1)^2 = \frac{1}{3}(4k^3 - k) + (2(k+1)-1)^2
  \]

  \text{Factor:}
  \[
    \frac{1}{3}(4k^3 - k) + (2(k+1)-1)^2 = \frac{1/3}(4k^3 - k) + (2k + 2 - 1)^2
  \]

  \text{Simplify:}
  \[
    \frac{1}{3}(4k^3 - k) + (2k + 1)^2 = \frac{1}{3}(4k^3 - k) + 4k^2 + 4k + 1
  \]

  \text{Conclude:}
  By combining terms, we get
  \[
    \frac{1}{3}(4k^3 - k + 9k^2 + 12k + 3)
  \]

  Factor and simplify further:
  \[
    \frac{(k+1)(4k^2 + 5k + 3)}{3}
  \]

  Simplify:
  \[
    \frac{(k+1)(4k^2 + 5k + 3)}{3}
  \]

  Factor the quadratic expression:
  \[
    \frac{(k+1)(4k+3)(k+1)}{3}
  \]

  Simplify the right side to get
  \[
    \frac{4(k+1)(k+1)(k+1)}{3}
  \]
  By mathematical induction, we have shown that the sum of the squares of the
  first $(n)$ odd positive integers is given by $((1/3)*(4n^3 - n))$.
\end{solution}

\newpage
\begin{problem} (20 points) Section 4.1, Exercise 4.6
\end{problem}
\begin{solution}
  ~\\
  ~\\
  Base Case:
  For \(n = 1\), we have:
  \[
    22^1 - 1 = 22 - 1 = 21
  \]
  Since 21 is divisible by 3, the formula holds for the base case. Inductive
  Hypothesis: Assume that the formula holds for some positive integer \(k\):
  \[
    22^k - 1\text{ is divisible by 3.}
  \]
  Inductive Step: Prove: \(22^{k+1} - 1\) is divisible by 3. \text{Start with the
    left side and use the inductive hypothesis:}
  \[
    22^{k+1} - 1 = 22 \cdot 22^k - 1 = 21 \cdot 22^k + 22^k - 1
  \]
  \text{Factor out \(22^k\) from the first term:}
  \[
    21 \cdot 22^k + 22^k - 1 = 22^k(21 + 1) - 1
  \]
  \text{Simplify:}
  \[
    22^k(22) - 1 = 22(22^k) - 1
  \]
  \text{Since \(22^k - 1\) is divisible by 3 (by the inductive hypothesis), express it as \(3m\) for some positive integer \(m\):}
  \[
    22(22^k) - 1 = 22(3m) = 3(22m)
  \]
  \text{Since \(22m\) is also a positive integer, we can see that \(22^{k+1} - 1\) is divisible by 3.}
  ~\\
  By mathematical induction, we've shown that \(22^n - 1\) is divisible by 3 for all positive integers \(n\).
\end{solution}

\newpage
\begin{problem} (20 points) Section 4.3, Exercise 4.15
\end{problem}
\begin{solution}
  ~\\
  ~\\
  Base Case:
  For \(n = 1\), we have:
  \[
    f_2 = 1, \text{ and } f_3 - 1 = 2 - 1 = 1.
  \]
  So, the formula holds for the base case.

  Inductive Hypothesis: Assume: \(f_2 + f_4 + \ldots + f_{2k} = f_{2k+1} - 1\).

  Inductive Step: Prove: $f_2 + f_4 + \ldots + f_{2(k+1)} = f_{2(k+1)+1} - 1$.
  \text{Start with the left side and use the inductive hypothesis:}
  \[
    f_2 + f_4 + \ldots + f_{2k} + f_{2(k+1)} = f_{2k+1} - 1 + f_{2(k+1)}.
  \]
  \text{Use the Fibonacci sequence property:}
  \[
    f_{2k+1} - 1 + f_{2(k+1)} = f_{2k+1} + f_{2k} - 1.
  \]
  \text{Simplify:}
  \[
    f_{2k+1} + f_{2k} = f_{2k-1} + f_{2k-2} + f_{2k}.
  \]
  \text{Apply the Fibonacci sequence property:}
  \[
    f_{2k-1} + f_{2k-2} + f_{2k} = f_{2k-1} + f_{2k-1} - 1 = 2f_{2k-1} - 1.
  \]
  \text{Apply the Fibonacci sequence property again:}
  \[
    2f_{2k-1} = f_{2k}.
  \]
  \text{So, we have:}
  \[
    f_{2k} - 1 = f_{2k} - 1.
  \]
  This proves that the formula holds for $(k + 1)$. So by mathematical induction,
  we have shown that the sum of even Fibonacci numbers $(f_2 + f_4 + \ldots +
    f_{2n})$ is indeed equal to $f_{2n+1} - 1$.
\end{solution}

\newpage
\begin{problem} (20 points) Section 4.6, Exercise 4.31
\end{problem}
\begin{solution}
  ~\\
  ~\\
  Base Case:
  For \(n = 4\), we have:
  \[
    n = 4 \text{ or } n -3 = 1
  \]
  \[
    f_4 = 4(4-1)(4-2)f_1
  \]
  \[
    f_4 = 4 * 3 * 2 * 1 = 4!
  \]
  \[
    f_4 = 4!
  \]
  ~\\
  ~\\
  Inductive Hypothesis:
  Assume given relation is true for $n = k$
  \[
    f_k = k(k-1)(k-2)f_{k-3}
  \]
  \[
    f_{k-1} = (k-1)(k-2)(k-3)f_{k-4}
  \]
  all are true ~\\ ~\\ Inductive Step: To prove relation is true for $n=k+1$ or
  \[
    f_{k+1} = (k+1)(k)(k-1)f_{k-2}
  \]
  ~\\
  ~\\
  From:
  \[
    f_{k-2} = (k-2)(k-3)(k-4)f_{k-5}
  \]
  \[
    f_{k-5} = (k-5)(k-6)(k-7)f_{k-8}
  \]
  \[
    f_4 = 4(4-1)(4-2)f_1
  \]
  ~\\
  ~\\
  Putting all of the values in $f_{k+1} = (k+1)(k)(k-1)f_{k-2}$ proves
  \[
    f_{k+1} = (k+1)!
  \]
\end{solution}

\end{document}
