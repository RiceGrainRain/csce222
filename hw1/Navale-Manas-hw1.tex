\documentclass{article}
\usepackage{amsmath,amssymb,amsthm,latexsym,paralist}
\usepackage{fancyhdr}

\theoremstyle{definition}
\newtheorem{problem}{Problem}
\newtheorem*{solution}{Solution}
\newtheorem*{resources}{Resources}

\newcommand{\name}[2]{\noindent\textbf{Name: #1}\hfill \textbf{UIN: #2}
  \newcommand{\myName}{#1}
  \newcommand{\myUIN}{#2}
}
\newcommand{\honor}{\noindent On my honor, as an Aggie, I have neither
  given nor received any unauthorized aid on any portion of the
  academic work included in this assignment. Furthermore, I have
  disclosed all resources (people, books, web sites, etc.) that have
  been used to prepare this homework. \\[2ex]
 \textbf{Electronic signature:} \underline{ \textbf{Manas Navale} } } % Type your full name here
 
\newcommand{\checklist}{\noindent\textbf{Checklist:}
\begin{compactitem}[$\Box$] 
\item Did you type in your name and UIN? 
\item Did you disclose all resources that you have used? \\
(This includes all people, books, websites, etc.\ that you have consulted.)
\item Did you sign that you followed the Aggie Honor Code? 
\item Did you solve all problems? 
\item Did you submit both the .tex and .pdf files of your homework to each correct link on Canvas? 
\end{compactitem}
}

\newcommand{\problemset}[1]{\begin{center}\textbf{Problem Set #1}\end{center}}
\newcommand{\duedate}[1]{\begin{quote}\textbf{Due dates:} Electronic
    submission of \textsl{yourLastName-yourFirstName-hw1.tex} and 
    \textsl{yourLastName-yourFirstName-hw1.pdf} files of this homework is due on
    \textbf{#1} on \texttt{https://canvas.tamu.edu}. You will see two separate links
    to turn in the .tex file and the .pdf file separately. Please do not archive or compress the files.  
    \textbf{If any of the two files are missing, you will receive zero points for this homework.}\end{quote} }

\newcommand{\N}{\mathbf{N}}
\newcommand{\R}{\mathbf{R}}
\newcommand{\Z}{\mathbf{Z}}

\fancyhead[L]{\myName}
\fancyhead[R]{\myUIN}
\pagestyle{fancy}

\begin{document}
\begin{center}
{\large
CSCE 222 Discrete Structures for Computing -- Fall 2023\\[.5ex]
Hyunyoung Lee\\}
\end{center}
\problemset{1}
\duedate{Tuesday, 9/5/2023 11:59 p.m.}
\name{ Manas Navale }{ 333006797 }  % Type your name and UIN here
% Omit the parentheses surrounding name and UIN.
% Your name should include your first and last names. 
% Your name and UIN that you type in here are propagated by LaTeX 
% to the header part of each page on the PDF output automatically.

\begin{resources} (All people, books, articles, web pages, etc. that
  have been consulted when producing your answers to this homework)
\end{resources}
\honor

\bigskip

\noindent
Total $100$ points.

\bigskip

\noindent
The intended formatting is that this first page is a cover page and each 
problem solved on a new page. You only need to fill in your solution between 
the \verb|\begin{solution}| and \verb|\end{solution}| environment.  
Please do not change this overall formatting.

\vfill
\checklist

\newpage
\begin{problem} ($10+10=20$ points) Section 1.1, Exercise 1.3.
For (b), give the knight's graph in a text format by giving all
edges in the graph such that the knight's move from vertex $v_i$ to 
vertex $v_{i+1}$ is given as $(v_i, v_{i+1})$.  Once you have all of the
edges written, you can also give the path in the form of 
$v_i - v_{i+1} - v_{i+2} - \ldots$

Use the common convention of expressing the columns and rows of
a chessboard as a, b, and c, and 1, 2, and 3, respectively.
\end{problem}
\begin{solution}
  For part a, it is impossible for a knight to visit all 9 squares because of the way the knight moves. 
  Since the board is 3x3, even if the knight starts at a corner it can only go to an edge. 
  If it start at an edge it can only go to a corner. 
  And it it starts at a center square it can only go to a center.
  it is unable to make the moves necessary to visit all 9 squares.
  \\
  \\
  Part b:
  \\(a1, c2)
  \\(a1, b3)
  \\(b1, a3)
  \\(b1, c3)
  \\(c1, a2)
  \\(c1, b3)
  \\(a2, c1)
  \\(a2, c3)
  \\(b2, a1)
  \\(b2, a3)
  \\(b2, c1)
  \\(b2, c3)
  \\(c2, a1)
  \\(c2, a3)
  \\(c3, a2)
  \\(c3, b1)
  \\
  \\
  Path:
  \\a1 - c2 - a3 - c1 - a2 - c3 - b1 - a3 - b2 - c3 - b2 - a1 - c2 - a1 - b3 - c1 - b3
\end{solution}

\newpage
\begin{problem} (2 points $\times$ 5 subproblems = 10 points) Section 2.1, Exercise 2.1
\end{problem}
\begin{solution}
  (a) is false because Pi is an irrational number but not the smallest Eg. {$\sqrt{2}$}.
  \\(b) is true because Pi is an irrational number as it can't be expressed as a ratio of two integers.
  \\(c) is false, as the quadration equation does have solutions x={$2 -\sqrt{2}$}, x={$2 +\sqrt{2}$}
  \\(d) is false as 23 is not less than 23 and is actually equal to it.
  \\(e) is not a statement, as it depends on the context of what x is. 
  \\All of the previous options were statements because they could be proved true or false.
\end{solution}

\newpage
\begin{problem} (3 points $\times$ 5 subproblems = 15 points) Section 2.1, Exercise 2.3
\end{problem}
\begin{solution}
  (a) is true because if {$x = 0.111...$} and 1{$10x = 1.111...$} then if you subtract {$10x-x$} you get {$1$} which means that {$9x=1$} which means {$x=1/9$}
  \\(b) is false as 0.121212 can be expressed as the fraction 121/990.
  \\(c) is true because there are no other common factors that divide into 1111 and 11111111
  \\(d) is true becasue the product of two negatives is positive so {$(-1)(-1) = 1$}
  \\(e) is true since all positive whole numbers would continue indefinitely leading to infinity.
\end{solution}

\newpage
\begin{problem} (2 points $\times$ 2 subproblems = 4 points) Section 2.2, Exercise 2.7 (a) and (b)
\end{problem}
\begin{solution}
  (a) Albert cooks pasta, and Emmy is not happy
  \\(b) If Albert cooks pasta, then Albert is happy and Emmy is also happy.
\end{solution}

\newpage
\begin{problem} (3 points $\times$ 2 subproblems = 6 points) Section 2.2, Exercise 2.8 (a) and (d)
\end{problem}
\begin{solution}
  (a) {$C \rightarrow \neg S$}
  \\(d) {$S \leftrightarrow \neg C$}
\end{solution}

\newpage
\begin{problem} (15 points) Section 2.2, Exercise 2.18.
Use a truth table to show your reasoning. 

Example \LaTeX\ source for how to draw a truth table is shown 
in the truth-table.tex and truth-table.pdf files.
\end{problem}
\begin{solution}
  if A is a knight, then the statement is true, and if A is a knave the statement is false.
  But there is not enough information to detemine the status of B since there was no statement, 
  and as a result A's status is uncertain as well
  \begin{displaymath}
    \begin{array}{|c c||c|c|}
    p & q & p \land q & p \oplus q\\
    \hline
    \text{Knight} & \text{Knight} & \text{True} & \text{False}\\
    \text{Knight} & \text{Knave} & \text{True} & \text{False}\\
    \text{Knave} & \text{Knight} & \text{False} & \text{True}\\
    \text{Knave} & \text{Knave} & \text{False} & \text{True}\\
    \end{array}
    \end{displaymath}
\end{solution}

\newpage
\begin{problem} (10 points) Section 2.3, Exercise 2.25.
Use a truth table.
\end{problem}
\begin{solution} 
  As shown in the truth table $(A \rightarrow B) \land (B \rightarrow A)$ and $A \leftrightarrow B$ are the same for all combinations of A and B, showing that they are equivalent.
    \begin{displaymath}
      \begin{array}{|c|c||c|c|c|c|c|}
      \hline
      A & B & A \rightarrow B & B \rightarrow A & (A \rightarrow B) & (B \rightarrow A) & (A \rightarrow B) \land (B \rightarrow A) \\
      \hline
      T & T & T & T & T & T & T \\
      T & F & F & T & F & T & F \\
      F & T & T & F & T & F & F \\
      F & F & T & T & T & T & T \\
      \hline
      \end{array}
    \end{displaymath}  
\end{solution}

\newpage
\begin{problem} (20 points) Section 2.3, Exercise 2.26.
Your answer should consist of a series of logical equivalences 
you learned in the text, and the final step must resolve to $T$.
Do not use a truth table.  Study the proofs of Proposition 2.8 (b) and (c) 
for the expected style of your answer.  Watching the video ``Problem 
Solving Exercise 1" in Module {2.1} will also be helpful.

Example \LaTeX\ source for how to align the steps nicely is shown 
in the truth-table.tex and truth-table.pdf files.
\end{problem}
\begin{solution} 
  STEPS
  \\1. $(A \land (A \rightarrow B)) \rightarrow B$
  \\2. $(A \land (\neg A \lor B)) \rightarrow B$
  \\3. $((A \land \neg A) \lor (A \land B)) \rightarrow B$
  \\4. $(\text{False} \lor (A \land B)) \rightarrow B$
 \\ 5. $(A \land B) \rightarrow B$
  \\6. $B$

\end{solution}

\end{document}
