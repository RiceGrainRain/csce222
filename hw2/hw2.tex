\documentclass{article}
\usepackage{amsmath,amssymb,amsthm,latexsym,paralist}
\usepackage{fancyhdr}

\theoremstyle{definition}
\newtheorem{problem}{Problem}
\newtheorem*{solution}{Solution}
\newtheorem*{resources}{Resources}

\newcommand{\name}[2]{\noindent\textbf{Name: #1}\hfill \textbf{UIN: #2}
  \newcommand{\myName}{#1}
  \newcommand{\myUIN}{#2}
}
\newcommand{\honor}{\noindent On my honor, as an Aggie, I have neither
  given nor received any unauthorized aid on any portion of the
  academic work included in this assignment. Furthermore, I have
  disclosed all resources (people, books, web sites, etc.) that have
  been used to prepare this homework. \\[2ex]
 \textbf{Electronic signature:} \underline{ \textbf{(Manas Navale)} } } % type your full name here
 
\newcommand{\checklist}{\noindent\textbf{Checklist:}
\begin{compactitem}[$\Box$] 
\item Did you type in your name and UIN? 
\item Did you disclose all resources that you have used? \\
(This includes all people, books, websites, etc.\ that you have consulted.)
\item Did you sign that you followed the Aggie Honor Code? 
\item Did you solve all problems? 
\item Did you submit both the .tex and .pdf files of your homework to each correct link on Canvas? 
\end{compactitem}
}

\newcommand{\problemset}[1]{\begin{center}\textbf{Problem Set #1}\end{center}}
\newcommand{\duedate}[1]{\begin{quote}\textbf{Due dates:} Electronic
    submission of \textsl{yourLastName-yourFirstName-hw2.tex} and 
    \textsl{yourLastName-yourFirstName-hw2.pdf} files of this homework is due on
    \textbf{#1} on \texttt{https://canvas.tamu.edu}. You will see two separate links
    to turn in the .tex file and the .pdf file separately. Please do not archive or compress the files.  
    \textbf{If any of the two files are missing, you will receive zero points for this homework.}\end{quote} }

\newcommand{\N}{\mathbf{N}}
\newcommand{\R}{\mathbf{R}}
\newcommand{\Z}{\mathbf{Z}}

\fancyhead[L]{\myName}
\fancyhead[R]{\myUIN}
\pagestyle{fancy}

\begin{document}
\begin{center}
{\large
CSCE 222 Discrete Structures for Computing -- Fall 2023\\[.5ex]
Hyunyoung Lee\\}
\end{center}
\problemset{2}
\duedate{Monday, 9/18/2023 11:59 p.m.}
\name{ (Manas Navale) }{ (333006797) }  % Type your first and last name and UIN here
% Omit the parentheses surrounding name and UIN.
% Your name should include your first and last names. 
% Your name and UIN that you type in here are propagated by LaTeX 
% to the header part of each page on the PDF output automatically.

\begin{resources} (All people, books, articles, web pages, etc.\ that
  have been consulted when producing your answers to this homework)
\end{resources}
\honor

\bigskip

\noindent
Total $100$ points.

\bigskip

\noindent
The intended formatting is that this first page is a cover page and each 
problem solved on a new page. You only need to fill in your solution between 
the \verb|\begin{solution}| and \verb|\end{solution}| environment.  
Please do not change this overall formatting.

\vfill
\checklist

\newpage
\begin{problem} ($5+5=10$ points) Section 2.6, Exercise 2.53 (a) and (c). Explain.
\end{problem}
\begin{solution}
  ~\\
  ~\\
   (a) 0 values.
   \\(the equation $a^n+b^n = c^n$ has no values when n is $\geq 3$ )
   \\(b) 1 values.
   \\(3, 4, 5)
\end{solution}

\newpage
\begin{problem} ($5+5=10$ points) Section 2.6, Exercise 2.54 (b) and (c)
\end{problem}
\begin{solution} 
  ~\\
  ~\\
  (b)For every value of x, there exists a value of y such that x is less than y.
  \\(c) For all values of x and for all values of z, there exists a value of y such that if x is less than z, then both x is less than y and y is less than z.
\end{solution}

\newpage
\begin{problem} ($5+5=10$ points) Section 2.7, Exercise 2.58 (a) and (e)
\end{problem}
\begin{solution} 
  ~\\
  (a)($\forall x \exists y (P(x) \land \neg Q(y))$)
  \\(e)($\forall x \forall y (P(x) \lor Q(y))$)
\end{solution}

\newpage
\begin{problem} ($5+5=10$ points) Section 2.7, Exercise 2.59 (d) and (e)
\end{problem}
\begin{solution} 
  ~\\
  (d)For all integers a, there exists an integer b such that ($a + b \neq 1001$).
  \\(e)There exists a positive integer a such that for all positive integers b, ($b \geq a$).
\end{solution}

\newpage
\begin{problem} (15 points) Section 2.9, Exercise 2.73
[Hint: Use the property of ``consecutive integers" and the definition of an ``odd integer".]
\end{problem}
\begin{solution} 
  ~\\
  \\($ n = m + 1$)
  \\($ m + n = m + (m + 1)$)
  \\($ m + n = 2m + 1$)
  \\Since m is an integer, 2m + 1 is also an integer, and an odd integer is an integer that can be expressed in the form ($2m + 1$)
\end{solution}

\newpage
\begin{problem} (15 points) Section 2.9, Exercise 2.80 
\end{problem}
\begin{solution}
  ~\\
  \\if ($m + n > 100$), then ($m > 40$) or ($n > 60$)
  \\The negation of this is ($m \leq 40$) and ($ n \leq 60$)
  \\to prove it by contraposition we need to show the negation implies the negation of the original statement.
  \\So ($m + n > 100$) is false which means ($m + n \leq 100$).
  \\Since ($m \leq 40$) and ($ n \leq 60$) then we can combine the two inequalities ($m + n \leq 100$)
  \\So if ($m \leq 40$) and ($ n \leq 60$) then ($m + n \leq 100$) which is the negation of the original statement.
  \\So by contraposition this means that ($m + n > 100$), and ($m > 40$) or ($n > 60$)
\end{solution}

\newpage
\begin{problem} (15 points) Section 2.9, Exercise 2.84 
\end{problem}
\begin{solution} 
  ~\\
  \\simplifying the equation ($42m + 70n = 1000$) to ($3m + 5n = 71$)
  ~\\
  \\If both 3m and 5n leave a remainder of 2 when divided by their respective divisors, then 3m+5n would leave a remainder of 4 when divided by 3. But 71 divided by 3 leaves a remainder of 2, which is a contradiction.
\end{solution}

\newpage
\begin{problem} (15 points) Section 3.3, Exercise 3.20 
[Hint: Use the definitions of $\subseteq$, $\cup$, and the power set.]
\end{problem}
\begin{solution}
~\\
  \\Using $X$ be an arbitrary element in $P(A) \cup P(B)$. This implies that $X$ is an element of either $P(A)$ or $P(B)$. 
  \\Using $X \in P(A)$, which means $X$ is a subset of $A$
  \\Using $X \in P(B)$, which means $X$ is a subset of $B$.
  \\Now, consider the union $A \cup B$. Any subset of $A \cup B$ contains elements from both $A$ and $B$. 
  \\Since $X$ is a subset of either $A$ or $B$ it is also a subset of $A \cup B$ because $A$ and $B$ are both subsets of $A \cup B$. Therefore, $X$ is an element of $P(A \cup B)$.
  \\So we have shown that any arbitrary element $X$ in $P(A) \cup P(B)$ is also an element of $P(A \cup B)$. Thus, $P(A) \cup P(B) \subseteq P(A \cup B)$.
\end{solution}

\end{document}
